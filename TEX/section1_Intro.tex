
\section{Introduction}
\par

Shallow Water Equations model the propagation of disturbances in water and other incompressible fluids and are used to describe the 
dynamics of important phenomenon like tsunami. The underlying assumption is that the depth of the fluid is small compared to 
the wave length of the disturbance. The conservative form of the shallow water equations is,

\begin{equation}\label{eqn:1}
{\frac{\partial h}{\partial t} + \frac{\partial (hu)}{\partial x} + \frac{\partial (hv)}{\partial y} = 0 } \notag
\end{equation}
\begin{equation}
{\frac{\partial (hu)}{\partial t} + \frac{\partial (hu^2 + \frac{1}{2} gh ^2)}{\partial x} + \frac{\partial (huv)}{\partial y} = fhv }
\end{equation}
\begin{equation}
{\frac{\partial (hv)}{\partial t} + \frac{\partial (huv)}{\partial x}+\frac{\partial (hv^2 + \frac{1}{2} gh ^2)}{\partial y}=-fhu } \notag
\end{equation}

Here $h$ $\ge$ 0 is the fluid height, $u$ and $v$ are the horizontal and vertical velocities, $g$ is the acceleration due to gravity ($9.8m/s^2$ on Earth) 
and $f$ is the Coriolis force. We let


\[U = \begin{bmatrix} h \\ hu \\ hv\end{bmatrix},\,\, F(U) = \begin{bmatrix} hu \\ hu^2 +  \frac{1}{2} gh^2 \\ huv \end{bmatrix},\,\,
G(U) = \begin{bmatrix} hv \\ huv \\ hv2 +  \frac{1}{2} gh^2 \end{bmatrix},\,\, 
S(U) = \begin{bmatrix} 0 \\ fhv \\ -fhu \end{bmatrix}, \]


which can now be rewritten in a more compact form,
\begin{equation}
 {\frac{\partial U}{\partial t} + \frac{\partial F(U)}{\partial x} + \frac{\partial G(U)}{\partial y} = S(U) } \label{eqn:2}
 \end{equation}

 
In the absence of the Coriolis force, we get the standard form of the conservation law,
\begin{equation}
{\frac{\partial U}{\partial t} + \frac{\partial F(U)}{\partial x} + \frac{\partial G(U)}{\partial y} = 0 } \label{eqn:3}
\end{equation}
We specify either periodic boundary condition, or ``free boundary'' conditions for $h$ and ``reflective'' boundary conditions for $uh$ and $uv$.
Free boundary conditions mean the boundary exerts no stress, while reflective boundary conditions mean the boundary behaves like a mirror.




