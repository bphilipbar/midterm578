\documentclass{article}

\usepackage{graphicx}
\usepackage{amsmath}
\usepackage{color}
\usepackage{breqn}
\usepackage{enumitem}
\usepackage{caption}
\usepackage{subcaption}
\usepackage[makeroom]{cancel}


\makeatletter
\newcommand\incircbin
{%
  \mathpalette\@incircbin
}
\newcommand\@incircbin[2]
{%
  \mathbin%
  {%
    \ooalign{\hidewidth$#1#2$\hidewidth\crcr$#1\bigcirc$}%
  }%
}
\newcommand{\oeq}{\incircbin{=}}
\newcommand{\ole}{\incircbin{\le}}
\newcommand\Ccancel[2][black]{\renewcommand\CancelColor{\color{#1}}\cancel{#2}}

\makeatother


\title{Homework 1}
\author{Oleksii Beznosov}
\date{\today}

\begin{document}
\maketitle
\begin{flushleft}

\textbf {Problem 1.}
\par Consider the linear first order PDE,
\begin{equation}\label{eq1}
u_t - a u_x = 0,\,\,\,\,0 < x < 1,\,\,\,\, 0 < t < 1,
\end{equation}
\par where $a > 0$. Use the method of characteristics to derive 
approriate initial and boundary conditions.
\par
\textbf {Solution.}
\par
Assume that $u$ is the solution to the equation \eqref{eq1}. And let
$\Gamma (x_0) = \{(\xi(t), t)\,\, | \,\,\xi(0) = x_0,\,\, 0 < t < 1\},\,\, 0 < x_0 < 1$,
be a curve. Define $h(t) = u(\xi(t), t)$ to be a
solution along $\Gamma(x_0)$. We have:

\begin{equation}\label{eq2}
\frac{dh}{dt} = u_t + u_x \frac{d\xi}{dt}.  
\end{equation}

Assume now that $h(t)$ is a constant and using the equation \eqref{eq1}
we get,

\begin{equation}\label{eq3}
\frac{dh}{dt} = a u_x + u_t \frac{d\xi}{dt} = 0, \forall t \in (0,1).
\end{equation}

Hence $\frac{d\xi}{dt} = -a$, and using initial condition for $\xi$ we
get 

\begin{equation}\label{eq4}
\xi(t) = x_0-at.
\end{equation}
Since $u$ is a constant along the curve $\Gamma$ we will use the initial
condition in order to define the solution. Suppose we have the initial 
condition at $t=0$, $u(x,0) = u_0(x)$. Then 

\[u(\xi(t),t) = u_0(x_0) = u_0(\xi(t) + at),\]
So solution can be written as $u(x,t) = u_0(x + at)$. In addition we
can derive boundary conditions the left in order to have all characteristics in the domain  
$u(1,t)  = u_0(1 + at).$ 
\newpage
\par \textbf {Problem 2.}
\par Show that the truncation error for the Crank-Nicolson method for the
heat equation, $u_t = u_{xx}$ is $O(\Delta x^2 + \Delta t^2)$.
\par \textbf {Solution.}
\par 
Consider Crank-Nicolson scheme for the heat equation,

\begin{equation}\label{eqCNS}
\frac{U_j^{n+1} - U_j^{n}}{\Delta t} = \frac{D^2 U_j^{n+1} + D^2 U_j^{n}}{2}. 
\end{equation}
Assume that $u(x,t)$ is the solution and plug in it into the scheme. Truncation error
T going to be a difference between right hand side and left hand side.
\begin{equation}\label{eqCNS_SOL}
\begin{split}
\frac{u(x,t + \Delta t) - u(x,t)}{\Delta t} & \approx 
\frac{u(x - \Delta x,t + \Delta t) - 2u(x,t+ \Delta t) + u(x + \Delta x,t+ \Delta t)}{2\Delta x^2} + \\
& +\frac{u(x - \Delta x,t) - 2u(x,t) + u(x + \Delta x,t)}{2\Delta x^2}. \\  
\end{split}
\end{equation}
For the simplicity we can work on the parts of \eqref{eqCNS_SOL} independently first using Taylor series. 

\begin{dmath}\label{eqCNS_LHS}
\frac{u(x,t + \Delta t) - u(x,t)}{\Delta t} = \frac{u(x,t) + \Delta t u_t(x,t) + \frac{\Delta t ^ 2}{2}u_{tt}(x,t) + \frac{\Delta t ^ 3}{6}u_{ttt}(x,t) + ... - u(x,t)}{\Delta t} =
\frac{\Delta t u_t(x,t) + \frac{\Delta t ^ 2}{2}u_{tt}(x,t) + \frac{\Delta t ^ 3}{6}u_{ttt}(x,t) + \frac{\Delta t ^ 4}{24}u_{tttt}(x,t) + ...}{\Delta t} =
u_t(x,t) + \frac{\Delta t}{2}u_{tt}(x,t) + \frac{\Delta t^2}{6}u_{ttt}(x,t) + \frac{\Delta t ^ 3}{24}u_{tttt}(x,t) + ... 
\end{dmath}

\begin{dmath}\label{eqCNS_RHS}
\frac{u(x - \Delta x,t) - 2 u(x,t) + u(x + \Delta x,t)}{2 \Delta x^2} = 
  \frac{\cancel{u(x,t)} - \Ccancel[blue]{\Delta x u_x(x,t)} + \frac{\Ccancel[red]{\Delta x ^ 2}}{2}u_{xx}(x,t) - \Ccancel[blue]{\frac{\Delta x^3}{6}u_{xxx}(x,t)} + ... }{2 \Ccancel[red]{\Delta x^2}}
+ \frac{\cancel{u(x,t)} + \Ccancel[blue]{\Delta x u_x(x,t)} + \frac{\Ccancel[red]{\Delta x ^ 2}}{2}u_{xx}(x,t) + \Ccancel[blue]{\frac{\Delta x^3}{6}u_{xxx}(x,t)} + ... - \cancel{2 u(x,t)}}{2\Ccancel[red]{\Delta x^2}}
= \frac{1}{2} u_{xx}(x,t) + \frac{\Delta x ^2}{24} u_{xxxx}(x,t) + ... 
= \frac{1}{2} u_{t}(x,t) + \frac{\Delta x ^2}{24} u_{tt}(x,t) + ... 
\end{dmath}

\par The last equation comes from the heat equation since $u_t = u_{xx}$.

\begin{dmath}\label{eqCNS_RHS_T}
\frac{u(x - \Delta x,t+\Delta t) - 2 u(x,t+\Delta t) + u(x + \Delta x,t+\Delta t)}{2 \Delta x^2} = 
  \frac{\cancel{u(x,t+\Delta t)} - \Ccancel[blue]{\Delta x u_x(x,t+\Delta t)} + \frac{\Ccancel[red]{\Delta x ^ 2}}{2}u_{xx}(x,t+\Delta t) - \Ccancel[blue]{\frac{\Delta x^3}{6}u_{xxx}(x,t+\Delta t)} + ... }{2 \Ccancel[red]{\Delta x^2}}
+ \frac{\cancel{u(x,t+\Delta t)} + \Ccancel[blue]{\Delta x u_x(x,t+\Delta t)} + \frac{\Ccancel[red]{\Delta x ^ 2}}{2}u_{xx}(x,t+\Delta t) + \Ccancel[blue]{\frac{\Delta x^3}{6}u_{xxx}(x,t+\Delta t)} + ... - \cancel{2 u(x,t+\Delta t)}}{2\Ccancel[red]{\Delta x^2}}
= \frac{1}{2} u_{xx}(x,t+\Delta t) + \frac{\Delta x ^2}{24} u_{xxxx}(x,t + \Delta t) + ... \, \oeq
\end{dmath}

\par Now we will use equation to change partial derivatives $u_{xx}\to u_t$

\begin{dmath}\label{eqCNS_RHS_T1}
\oeq \, \frac{1}{2} u_{t}(x,t+\Delta t) + \frac{\Delta x ^2}{24} u_{tt}(x,t + \Delta t) + ... 
= \frac{1}{2} u_{t}(x,t) + \frac{\Delta t}{2} u_{tt}(x,t) + \frac{\Delta t^2}{4} u_{ttt}(x,t) + ... + \frac{\Delta x ^2}{24} u_{tt}(x,t) + \frac{\Delta x ^2 \Delta t}{24} u_{ttt}(x,t) + ... .  
\end{dmath}

Now collect everything:
\begin{dmath}\label{eqCNT}
T(x,t) = \frac{u(x,t + \Delta t) - u(x,t)}{\Delta t}
- \frac{u(x - \Delta x,t + \Delta t) - 2u(x,t+ \Delta t) + u(x + \Delta x,t+ \Delta t)}{2\Delta x^2} + \\
- \frac{u(x - \Delta x,t) - 2u(x,t) + u(x + \Delta x,t)}{2\Delta x^2}.
= \cancel{u_t(x,t)} + \Ccancel[blue]{\frac{\Delta t}{2}u_{tt}(x,t)} + \underline{\frac{\Delta t^2}{6}u_{ttt}(x,t)} + \frac{\Delta t ^ 3}{24}u_{tttt}(x,t) + ...
- \cancel{\frac{1}{2} u_{t}(x,t)} - \Ccancel[blue]{\frac{\Delta t}{2} u_{tt}(x,t)} - \underline{\frac{\Delta t^2}{4} u_{ttt}(x,t)} - ... - \underline{\frac{\Delta x ^2}{24} u_{tt}(x,t)} - \frac{\Delta x ^2 \Delta t}{24} u_{ttt}(x,t) - ...
- \cancel{\frac{1}{2} u_{t}(x,t)} - \underline{\frac{\Delta x ^2}{24} u_{tt}(x,t)} - ...
\end{dmath}
Since $u(x,t)$ is smooth (so has all derivatives bounded) there exists $M$ large enough such that
\begin{dmath}\label{eqCNT11}
|T(x,t)| \le M(\Delta t^2 + \Delta x^2).
\end{dmath}
Hence, $T(x,t) = O(\Delta t^2 + \Delta x^2)$.

\newpage
\par \textbf {Problem 3.}
\par
Consider the following numerical scheme for the solution of heat equation,
$u_t = u_{xx}$ with homogeneous Dirichlet boundary conditions and
smooth initial conditions $\eta(x)$,
\begin{equation}\label{eqFS}
\frac{U_j^{n+1} - U_j^{n}}{\Delta t} = \frac{U_{j-1}^{n+1} -2 U_{j}^{n+1} + U_{j+1}^{n+1}}{\Delta x^2}. 
\end{equation}

\begin{enumerate}[label=(\alph*)]
\item Draw the stencil for this scheme. Is this an explicit or an implicit.
\item Derive the truncation error for this scheme.
\item Discuss the stability of this scheme by using Von Neumann analysis.
\end{enumerate}
\par \textbf {Solution.}
\par (a)
\begin{figure}[ht]
\begin{center}
\setlength{\unitlength}{1.0cm} 
\begin{picture}(8,2) 

\thicklines
\put(2,2){\line(2,0){4}}
\put(4,2){\line(0,-2){1}}
\multiput(2,2)(2,0){3}{\color{black}{\circle*{0.1}}}
\put(1.3,2.3){\color{black}{\scriptsize $(x_{j-1},t_{n+1}$)}}
\put(3.5,2.3){\color{black}{\scriptsize $(x_{j},t_{n+1}$)}}
\put(5.3,2.3){\color{black}{\scriptsize $(x_{j+1},t_{n+1}$)}}
\put(4,1){\color{black}{\circle*{0.1}}}2
\put(3.6,0.6){\color{black}{\scriptsize $(x_{j},t_{n}$)}}

\end{picture}
\caption{Stencil for the scheme \label{fig:Stensil1}}
\end{center}
\end{figure}

We see that $U_{j}^{n+1}$ depends on $U_{j+1}^{n+1}$ and $U_{j-1}^{n+1}$ so this scheme is implicit.

\par (b) Same as in previous problem. Assuming that $u(x,t)$ is the solution we have:
\begin{dmath}\label{eqFD_LHS}
\frac{u(x,t + \Delta t) - u(x,t)}{\Delta t} = \frac{u(x,t) + \Delta t u_t(x,t) + \frac{\Delta t ^ 2}{2}u_{tt}(x,t) + \frac{\Delta t ^ 3}{6}u_{ttt}(x,t) + ... - u(x,t)}{\Delta t} =
\frac{\Delta t u_t(x,t) + \frac{\Delta t ^ 2}{2}u_{tt}(x,t) + \frac{\Delta t ^ 3}{6}u_{ttt}(x,t) + \frac{\Delta t ^ 4}{24}u_{tttt}(x,t) + ...}{\Delta t} =
u_t(x,t) + \frac{\Delta t}{2}u_{tt}(x,t) + \frac{\Delta t^2}{6}u_{ttt}(x,t) + \frac{\Delta t ^ 3}{24}u_{tttt}(x,t) + ... 
\end{dmath}

\begin{dmath}\label{eqFD_RHS_T}
\frac{u(x - \Delta x,t+\Delta t) - 2 u(x,t+\Delta t) + u(x + \Delta x,t+\Delta t)}{\Delta x^2} = 
  \frac{\cancel{u(x,t+\Delta t)} - \Ccancel[blue]{\Delta x u_x(x,t+\Delta t)} + \frac{\Ccancel[red]{\Delta x ^ 2}}{2}u_{xx}(x,t+\Delta t) - \Ccancel[blue]{\frac{\Delta x^3}{6}u_{xxx}(x,t+\Delta t)} + ... }{\Ccancel[red]{\Delta x^2}}
+ \frac{\cancel{u(x,t+\Delta t)} + \Ccancel[blue]{\Delta x u_x(x,t+\Delta t)} + \frac{\Ccancel[red]{\Delta x ^ 2}}{2}u_{xx}(x,t+\Delta t) + \Ccancel[blue]{\frac{\Delta x^3}{6}u_{xxx}(x,t+\Delta t)} + ... - \cancel{2 u(x,t+\Delta t)}}{\Ccancel[red]{\Delta x^2}}
= u_{xx}(x,t+\Delta t) + \frac{\Delta x ^2}{12} u_{xxxx}(x,t + \Delta t) + ... 
= u_{t}(x,t+\Delta t) + \frac{\Delta x ^2}{12} u_{tt}(x,t + \Delta t) + ... 
= u_{t}(x,t) + \Delta t u_{tt}(x,t) + \frac{\Delta t^2}{2} u_{ttt}(x,t) + ... + \frac{\Delta x ^2}{12} u_{tt}(x,t) + \frac{\Delta x ^2 \Delta t}{12} u_{ttt}(x,t) + ... .  
\end{dmath}

\par Now we can estimate the truncation error

\begin{dmath}\label{eqFDT}
T(x,y) = \frac{u(x,t + \Delta t) - u(x,t)}{\Delta t} - \frac{u(x - \Delta x,t+\Delta t) - 2 u(x,t+\Delta t) + u(x + \Delta x,t+\Delta t)}{\Delta x^2}
=\cancel{u_t(x,t)} + \underline{\frac{\Delta t}{2}u_{tt}(x,t)} + \frac{\Delta t^2}{6}u_{ttt}(x,t) + \frac{\Delta t ^ 3}{24}u_{tttt}(x,t) + ... 
-\cancel{u_{t}(x,t)} - \underline{\Delta t u_{tt}(x,t)} - \frac{\Delta t^2}{2} u_{ttt}(x,t) - ... - \underline{\frac{\Delta x ^2}{12} u_{tt}(x,t)}
- \frac{\Delta x ^2 \Delta t}{12} u_{ttt}(x,t) + ...
\end{dmath}

Since $u(x,t)$ is assumed to be smooth (so has all derivatives bounded) there exists $M$ large enough such that
\begin{dmath}\label{eqFDT1}
|T(x,t)| \le M(\Delta t + \Delta x^2).
\end{dmath}
(c) Let $r = \frac{\Delta t}{\Delta x^2} $ and $U_j^{n+1} = g(r)e^{\mathbf{i}j \Delta x}$.
Then from the scheme \eqref{eqFS} we get:
\begin{dmath}\label{eqFDVN}
U^n_j = U^{n+1}_j - r (U^{n+1}_{j-1} - 2 U^{n+1}_{j} + U^{n+1}_{j+1}) 
= ge^{\mathbf{i}j \Delta x} - r (ge^{\mathbf{i}(j-1) \Delta x} - 2 ge^{\mathbf{i}j \Delta x} + ge^{\mathbf{i}(j+1) \Delta x})
= ge^{\mathbf{i}j \Delta x}[1 - r(ge^{-\mathbf{i} \Delta x} - 2 + ge^{\mathbf{i}\Delta x})]
= ge^{\mathbf{i}j \Delta x}[1 + 4r \sin^2(\Delta x)] 
= U_j^{n+1}[1 + 4r \sin^2(\Delta x)] 
\end{dmath}
Now we see
\begin{equation}\label{eqFDVE}
0 < \frac{U_j^{n+1}}{U_j^n} = \frac{1}{1 + 4r \sin^2(\Delta x)} < 1.
\end{equation}
Hence the scheme is unconditionally stable.

\par \textbf{Problem 4.}
Consider the heat equation $u_t = u_{xx}$ on the domain $x \in [0, 1]$ and
$t \in [0, tF ]$.

\begin{enumerate}[label=(\alph*)]
\item Write computer programs to solve the above equation using:
\begin{itemize}
\item The explicit scheme studied in class, that is, forward difference
in time, centered difference in space. (The code for this
scheme is given to you!)
\item Crank-Nicolson method
\end{itemize}
\item Assume homogeneous Dirichlet boundary conditions, $tF = 2$ and
the initial conditions given by $\nu(x) = \sin(2 \pi x)$. Test the stability
of your algorithms by trying a few different values of $\Delta x$ and $\Delta t$.
Be sure to include values above and below (but near) the stability
threshold for the explicit scheme. For each $\Delta t$, $\Delta x$ pair, make
a 2D plot of the solution for different values of $t$ (that is, each
figure is a plot $U(x, t_n)$ against $x$ for a fixed $t_n$) .
\item What is the analytic solution for the initial conditions given in
the previous part? Test the accuracy of your algorithm. Plot the
maximum error in your solution as you vary $\Delta t$ and $\Delta x$. Do you
observe the theoretical rate of convergence?
\item Experiment with the problem by adding time-dependent boundary
conditions at $x = 0$ given by $g_0(t) = b \sin(\omega t)$ for your choice
of $b$ and $\omega$. Plot the results in space-time as a 3D surface.
\end{enumerate}


\begin{figure}
    \centering
    \begin{subfigure}[t]{0.5\textwidth}
        \includegraphics[height=2.0in]{img/FDsol6_80}
        \caption{$\Delta x = \frac{1}{6}$, $\Delta t = \frac{1}{40}$}
        \label{fig:gull}
    \end{subfigure}
    %add desired spacing between images, e. g. ~, \quad, \qquad, \hfill etc. 
      %(or a blank line to force the subfigure onto a new line)
    \begin{subfigure}[t]{0.4\textwidth}
        \includegraphics[height=2.0in]{img/FDsol6_90f}
        \caption{$\Delta x = \frac{1}{6}$, $\Delta t = \frac{1}{45}$  }
        \label{fig:tiger}
    \end{subfigure}
    \caption{Forward Euler expilicit}\label{fig:animals}
\end{figure}

\begin{figure}
    \centering
    \begin{subfigure}[t]{0.5\textwidth}
        \includegraphics[height=2.0in]{img/CNsol6_30f}
        \caption{$\Delta x = \frac{1}{6}$, $\Delta t = \frac{1}{15}$}
        \label{fig:gull}
    \end{subfigure}
    %add desired spacing between images, e. g. ~, \quad, \qquad, \hfill etc. 
      %(or a blank line to force the subfigure onto a new line)
    \begin{subfigure}[t]{0.4\textwidth}
        \includegraphics[height=2.0in]{img/CNsol6_80}
        \caption{$\Delta x = \frac{1}{6}$, $\Delta t = \frac{1}{40}$  }
        \label{fig:tiger}
    \end{subfigure}
    \caption{Crank-Nicolson scheme}\label{fig:animals}
\end{figure}

\begin{figure}
    \centering
    \begin{subfigure}[t]{0.5\textwidth}
        \includegraphics[height=2.0in]{img/CNsol6_70f}
        \caption{$\Delta x = \frac{1}{6}$, $\Delta t = \frac{1}{35}$}
        \label{fig:gull}
    \end{subfigure}
    %add desired spacing between images, e. g. ~, \quad, \qquad, \hfill etc. 
      %(or a blank line to force the subfigure onto a new line)
    \begin{subfigure}[t]{0.4\textwidth}
        \includegraphics[height=2.0in]{img/CNsol20_10}
        \caption{$\Delta x = \frac{1}{20}$, $\Delta t = \frac{1}{5}$  }
        \label{fig:tiger}
    \end{subfigure}
    \caption{Crank-Nicolson scheme}\label{fig:animals}
\end{figure}

The exact solution is  $u(x,t) = e^{-4\pi^2 t}\sin(2\pi x)$

\begin{figure}
    \centering
    \begin{subfigure}[t]{0.5\textwidth}
        \includegraphics[height=2.0in]{img/solbce}
        \caption{Forward Euler in time}
        \label{fig:gull}
    \end{subfigure}
    %add desired spacing between images, e. g. ~, \quad, \qquad, \hfill etc. 
      %(or a blank line to force the subfigure onto a new line)
    \begin{subfigure}[t]{0.4\textwidth}
        \includegraphics[height=2.0in]{img/solbc}
        \caption{Solution with CN scheme}
        \label{fig:tiger}
    \end{subfigure}
    \caption{Solution with boundary condition $g(t) = 2\sin{4t}$}\label{fig:animals}
\end{figure}

As we see solution blow up for Forward Euler in time scheme when the stability condition is not 
satisfied. On the other hand Crank-Nicolson scheme is stable but gives physically unreasonable solution
(it chenges sign for a fixed x in through time) if we pick timestep too large.

\end{flushleft}
\end{document}

